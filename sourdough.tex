\documentclass[a4paper,12pt]{article}
\usepackage{mathpazo}
% Encoding and language setup
\usepackage[utf8]{inputenc} % UTF-8 encoding
\usepackage[T1]{fontenc} % Font encoding
\usepackage[slovak]{babel} % Slovak language support

% Useful packages
\usepackage{tcolorbox} % For styled boxes
\usepackage{graphicx} % For images
\usepackage{geometry} % Page layout
\geometry{margin=1in}

% Title and Author
\title{Domáci kváskový ražný handmade BIO chlieb}
\date{2024}

\begin{document}

\maketitle

% Table of contents
%\tableofcontents
%\newpage % Add a page break after the table of contents

\section{Úvod}
Táto príručka je výsledok zbierky všetkých diskusií, správ, mailov a iných materiálov, ktoré som za rok pečenia chleba zhromaždil. Vo výsledku je to ale len jeden z mnohých spôsobov, ako upiecť fenomenálny chlieb. Ak existuje k niektorému kroku alternatíva, budem sa snažiť na ňu upozorniť, prípadne spomenúť či s ňou mám osobnú skusenosť.

\section{Starostlivosť o kvások}
Najprv pár slov k tomu, ako s kváskom pracovať a udržovať ho pri živote. V zásade sa dokrmuje múkou a vodou v pomere 1:1. Ja používam ražnú múku, ale bežne sa používa aj pšeničná. Okrem malého rozdielu v chuti treba počítať s tým, že ražná múka trochu horšie kysne. Ale pokiaľ sa používa len na kvások, je to príliš malé množstvo aby na tom záležalo. Okrem toho ražná múka viac nasiakne vodou, preto dávam na 10g ražnej múky cca 11-12ml vody. Pri pšeničnom je to 1:1.

Obecne platí, že toľko kvásku sa odoberie, toľko sa pridá. Takže v mojom prípade odoberiem do chleba 125g\footnote{Neskôr v recepte budem písať 120g kvásku, pretože sa jedná o kvások miešaný v pomere 1:1. Ja pridávam trochu vody navyše, preto aj výslednej hmotnosti je o trochu viac.} a pridám 60g múky a 65ml vody. Nechám ho v izbovej teplote pár hodín s pootvoreným vrchnákom, aby sa rozbehol a potom ho odkladám uzavretý do chladničky.

Vydržať by mal aj týždeň, ale asi som to nikdy neskúšal.

\section{Príprava}

\subsubsection{Výber múky}
Výber múky má najväčší vplyv na to, ako bude vyzerať celý proces kysnutia, pečenia a vo výsledku aj samotný chlieb. Samozrejme, ideálna je chlebová múka, prekvapivo. Ale dôležité je všimnúť si obsah bielkovín v múke. Bežne sa v obchode predáva múka s obsahom bielkovín okolo 11-12\%. Najvyšší obsah má pšeničná múka na kváskovanie z mlyna Trenčan, 13,62\%. Aj tak malý rozdiel v množstve má veľký vplyv na pevnosť cesta. Chlieb sa dá piecť z každej múky, ale menej bielkovín znamená nižšiu hydratáciu cesta. Kombinácia trenčianskej múky a 80\% hydratácie je overená na viac ako stovke úspešne upečených chlebov.\footnote{70\% by ale s ostatnými múkami malo byt v pohode.} Bolo by fajn mať tabuľku aká maximálna hydratácia je vhodná na aký obsah bielkovín, ja ňou žiaľ nedisponujem.

Treba si uvedomiť, že rôzne zahraničné recepty veľmi pravdepodobne počítajú s inou múkou ako máme na slovensku. Napríklad chlebová múka v recepte z ameriky pravdepodobne bude mať okolo 15\% bielkovín, pretože ju vyrábajú z inej pšenice. \footnote{Hard red spring wheat}. Preto pozor pri pečení podľa iných receptov.

V kvásku je 60g ražnej múky a do cesta ide 600g pšeničnej. Jedná sa teda o 9\% ražný chlieb.


\subsubsection{Pšeníčná múka}
Pšeničná múka veľmi dobre kysne a má veľa bielkovín, výsledné cesto je pevné a dáva najväčšie predpoklady aby chlieb dobre vyzeral. Chýba jej ale výraznejšia chuť.
\subsubsection{Ražná múka}
Vraj je o niečo zdravšia, ale kto vie. Každopádne je lepšia na prikrmovanie kvásku a dodáva chlebu trochu inú chuť. Či je to lepšie je subjektívne. Je možné  nahradiť časť pšeničnej múky ražnou, ale znižuje sa schopnosť cesta dobre vykysnúť a tiež klesá percentuálny obsah bielkovín. Viac ako 25\% už podľa mňa je moc.

\subsection{Hydratácia cesta}
Hydratácia cesta sa označuje v percentách a vyjadruje množstvo vody v pomere k múke. Pre účely tohto receptu budem používať 70\% hydratáciu, pretože je to fajn stredná cesta. Príliš nízka hydratácia má za následok trochu suchší chlieb a nevydrží tak dlho čerstvý. Zato s cestom sa výborne pracuje a nelepí sa. Je možné začať s nižšou hydratáciou, ale pod 65\% by som už asi nešiel. Na druhej strane 80\% vody z cesta spraví pri spracovnávaní nočnú moru, cesto sa lepí stále a všade, ale výsledok stojí za to. Chlieb je mimoriadne nadýchaný a dlho vydrží čerstvý. Pri správnom skladovaní sa dá zjesť aj na 6. deň. Opakujem, chce to veľa trpezlivosti a dostať to trochu do rúk, lebo cesto miestami skoro tečie. Pomáha cesto schladiť, ale tým sa spomaľuje kysnutie, takže je to koniec koncov vždy na rozhodnutí pečúceho.

\section{Postup}
\begin{tcolorbox}[colframe=blue!50!black, colback=blue!5!white, title=Ingrediencie]
    \begin{itemize}
        \item 600g hladkej múky (prípadne podľa vlastnej preferencie)
        \item 420ml studenej vody
        \item 120g kvásku
        \item 12-18g soli

        Pre ľubovoľnú veľkosť chleba platí, že do cesta ide 100\% múky, 70\% vody, 20\% kvásku a  2-3\% soli.
    \end{itemize}
\end{tcolorbox}

Všetky ingrediencie zmiešať. Ideálne použiť kuchynský robot na max výkon 8-10 min. Dá sa to aj ručne a prekladať, ale s tým skúsenosť nemám. Vymiešané cesto treba zakryť mokrou utierkou, igelitom, včelobalom, čímkoľvek, aby neoschlo.

Cesto treba nechať vykysnúť cca 5-8 hodín. Veľmi záleží od teploty v miestnosti a možno aj iných okolností. Každopádne, cesto by malo zdvojnásobiť svoj objem.\footnote{Je dobrý zvyk si do malého pohára odobrať a poznačiť si, koľko je tam cesta, pretože v miske sa to ťažko odhaduje. Keď je v malom pohári 2x viac, znamená to, že aj zvyšok cesta je už hotový. Ale mne sa to nechce, tak to nerobím.}

Potom ho vyberiem na dosku, poriadne všetko pomúčim, lebo 70\% hydratované cesto už môže byť lepkavé. Zrolujem ho, potom znovu z druhej strany a snažím sa vyrobiť "bochník".

Do ošátky dám utierku a tú tiež smelo pomúčim. Prilepené cesto o utierku je katastrofa odlepiť. Potom do ošátky dám cesto a nechám ďalej kysnúť. 2h na linke by mali stačiť, ale radšej ho dávam do igelitovej tašky a do chladničky.\footnote{Keď to nedávam do chladničky, tak sa mi často prilepí na utierku. Tak radšej dávam do chladničky. Ale nie je to nutné.} Aspoň 4h, prípadne cez noc, ak idem piecť na druhý deň ráno. Ale tu už naozaj nesledujem hodiny a kľudne aj po 2 hodinách to môže byt v pohode.

Nasleduje pečenie. Doteraz som piekol vždy len v liatinovej zapekacej mise. Trúbu som aj s misou a pokrievkou vyhrial na maximum (v našom prípade 230) aspoň na pol hodinu. Keď už bola misa rozohriata, vyklopil som na pečiaci papier cesto z ošátky a celé to dal to misy. Papier s cestom som podlial 0,5dl studenej vody a rýchlo zakryl.\footnote{Pozor pri mise z jánskeho skla. Ak sa rozhorúči a naleje do toho studená voda, praskne. Liatinová je preto lepšia. A lepšie drží teplo.}

Pečiem 30min. Potom odokryjem vrchnák z misy a dopekám 15min alebo dokým nemá požadovanú kôrku.


\section{Záver}
Zlé jazyky vravia, že chlieb sa po vytiahnutí z trúby nesmie rozkrajovať minimálne hodinu, ideálne aj viac. Vraj spľasne, alebo bude človeku zle, iní vravia, že budete mať rok nešťastie. Realita je ale taká, že pre skutočne transcendentálny zážitok je treba chlieb rozkrojiť ideálne do 5 minút a ochutnať ho s maslom, prípadne šunkou.

\begin{tcolorbox}[colframe=green!50!black, colback=green!5!white, title=Tipy na Pečenie]
    \begin{itemize}
        \item Ked mi napadne....
    \end{itemize}
\end{tcolorbox}

\end{document}
